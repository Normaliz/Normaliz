\documentclass[12pt,a4paper]{scrartcl}
\usepackage[utf8]{inputenc}
\usepackage{amsmath}
\usepackage{amsfonts}
\usepackage{amsthm}
\usepackage{amssymb}
%\usepackage[utf8]{inputenc}
\usepackage{mathptmx}
\usepackage{array}
\usepackage{color}
\usepackage{hyperref}
\usepackage{url}
\usepackage{graphicx}
\usepackage{multicol}
\usepackage{dsfont}

\usepackage{makeidx}

\usepackage{booktabs}

%\usepackage[ignoreunlbld,nomsgs]{refcheck}

\usepackage{pgf, tikz}
\usepackage{color}

% !TeX spellcheck = en_US

%\textwidth=15cm \textheight=22cm \topmargin=0.5cm
%\oddsidemargin=0.5cm \evensidemargin=0.5cm

\usepackage[T1]{fontenc}

\usepackage[scaled=0.8]{beramono}

\usepackage{fancyvrb} \RecustomVerbatimEnvironment{Verbatim}{Verbatim}
{xleftmargin=15pt, frame=single, fontsize=\small}


\newcounter{listi}
\newcommand{\stdli}{ \topsep0ex \partopsep0ex % .5ex plus.25ex minus.125ex%
\parsep.2ex plus.1ex minus.1ex \itemsep0ex% .5ex plus.25ex minus.125ex%
\leftmargin2.5em \labelwidth2em \labelsep.5em \rightmargin0em}% \samepage }
\newenvironment{arab}{\begin{list}{\textup{(\arabic{listi})}}%
	{\usecounter{listi}\stdli}}{\end{list}}
\newenvironment{rome}{\begin{list}{\textup{(\roman{listi})}}%
	{\usecounter{listi}\stdli}}{\end{list}}
\newenvironment{latin}{\begin{list}{\textup{(\alph{listi})}}%
	{\usecounter{listi}\stdli}}{\end{list}}
\renewenvironment{itemize}{\begin{list}{{$\bullet$}}{\stdli}}{\end{list}}
\newenvironment{myverb}{\begin{small}}{\end{small}\pagebreak[2]}  %%%%%  \vspace{-0.8\baselineskip}


\newtheorem{theorem}{Theorem}
\newtheorem{lemma}[theorem]{Lemma}
\newtheorem{corollary}[theorem]{Corollary}
\newtheorem{proposition}[theorem]{Proposition}


\theoremstyle{definition}
\newtheorem{remark}[theorem]{Remark}
\newtheorem{definition}[theorem]{Definition}

\let\phi=\varphi

\def\CC{{\mathbb C}}
\def\ZZ{{\mathbb Z}}
\def\QQ{{\mathbb Q}}
\def\RR{{\mathbb R}}
\def\EE{{\mathbb E}}
\def\AA{{\mathbb A}}
\def\PP{{\mathbb P}}
\def\NN{{\mathbb N}}

\def\cL{{\mathcal L}}

\def\Ker{\operatorname{Ker}}
\def\Im{\operatorname{Im}}
\DeclareMathOperator{\gp}{gp}
\DeclareMathOperator{\rank}{rank}
\DeclareMathOperator{\conv}{conv}
\DeclareMathOperator{\aff}{aff}
\DeclareMathOperator{\cone}{cone}
\DeclareMathOperator{\rec}{rec}
\DeclareMathOperator{\mrank}{mrank}
\DeclareMathOperator{\Hilb}{Hilb}
\DeclareMathOperator{\vol}{vol}
\DeclareMathOperator{\Cl}{Cl}
\DeclareMathOperator{\para}{par}

\DeclareMathOperator{\totdeg}{totdeg}


\def\cG{{\mathcal G}}
\def\cR{{\mathcal R}}

\let\hat=\widehat
\let\tilde=\widetilde
\let\Bar=\overline

\let\iso=\cong

\let\epsilon=\varepsilon
\def\discuss#1{\marginparsep=1em\marginparwidth=60pt
\marginpar{\tt \footnotesize \raggedright #1}}

\definecolor{darkgray}{gray}{0.00}

\addtokomafont{section}{\color{darkgray}}

\setkomafont{sectionentry}{\large}

\addtokomafont{subsection}{\color{darkgray}}

\addtokomafont{subsubsection}{\normalsize}

\parindent=0pt \parskip=4pt

\setcounter{tocdepth}{3}

%\def\Normaliz#1+{\textsf{Normaliz}}
%\def\jNormaliz#1+{\textsf{jNormaliz}}

\def\itemtt[#1]{\index{#1@\texttt{#1}}\item[\textbf{\ttt{#1}}]}

\def\subsectionpage#1{\newpage\index{#1}\subsection{#1}}

\def\ttt{\texttt}

\def\version{3.9.2}
\def\NmzDir{normaliz-\version}

\def\ContLine#1{\hyperref[#1]{\large #1}\\[6pt]}
\def\SubSectStart#1{\subsection{#1}\label{#1}}
\def\SubSubSectStart#1{\subsubsection{#1}\label{#1}}

\def\SectStart#1{\section{#1}\label{#1}\def\SectHead{#1}}
\def\SectStart#1{\section{#1}\label{#1}\def\SectHead{#1}}

\renewcommand{\indexname}{Index\label{Index}}

%\def\EndOfPage#1{\bigskip\bigskip\par \hyperref[#1]{#1}\qquad\qquad\hyperref[Top]{Top}\qquad\qquad\hyperref[Index]{Index}\newpage}

\def\TopIndex{\bigskip\bigskip\par \hspace{120pt}\hyperref[Top]{\large Top}\hspace{120pt}\hyperref[Index]{\large Index}\newpage}

\def\EndOfPage{\bigskip\bigskip\par\centerline{ \hyperref[\SectHead]{\large\SectHead}\hspace{120pt}\hyperref[Top]{\large Top}\hspace{120pt}\hyperref[Index]{\large Index}}\newpage}

\hyphenation{semi-open}

\makeindex

\begin{document}
\vspace*{2cm}

\begin{center}
\LARGE  Under \textbf{construction !!}
\end{center}

\label{Top}
\begin{center}
\Large\textbf{Normaliz Short Reference}\\ Version \version
\end{center}

\ContLine{Generalities}
\ContLine{Command line}
\ContLine{Input}
\ContLine{Computation}
\medskip

\ContLine{Index}

\newpage

\SectStart{Generalities}

Normaliz is a tool for discrete convex geometry. It computes several data of polyhedra and lattice points in them.The names of the Normaliz input types and computation goals are descriptive and self explaining. We recommend the user to experiment with the examples in  the directory \verb|example|. A large part of the manual is in tutorial style.

Ways to run Normaliz:
\begin{enumerate}
\item in a terminal of Linux, MacOS or MS Windows,
\item in a Docker container (effectively a Linux terminal),
\item in the GUI interface jNormaliz,
\item interactively via the Python interface PyNormaliz (see Appendix E of the Normaliz manual),
\end{enumerate}
Moreover, Normaliz is used by several other packages, explicitly or implicitly. In this reference we assume that Normaliz is run in a terminal.

For a deeper understanding the one must note the following. The output depends to some extent on the types of input and of computation that can be \emph{homogeneous} or \emph{inhomogeneous}
\begin{enumerate}
\item Homogeneous input types define cones and lattices.

\item Inhomogeneous input types define polyhedra and an affine lattices.
\end{enumerate}
For the computation, inhomogeneous input is homogenized, and the polyhedron and the affine lattice are selected by the \emph{dehomogenization}. The computation is inhomogeneous if a dehomogenization is defined. Polytopes, i.e., bounded polyhedra, can be defined by homogeneous input if one adds a grading, or, as expected, by inhomogeneous input. These two approaches are almost equivalent.

\TopIndex


\SectStart{Command line}
\index{command line}

Normaliz is run in a terminal The format of the command line is
\begin{quote}
	\verb|normaliz [options]| <project>
\end{quote}
\verb|project| defines the input file \verb|<project.in|. The output files are \verb|<project>.<suffix>|. The main out put file is \verb|<project>.out|. Depending on your path settings and the place where \verb|normaliz| is installed you may have to prefix \verb|normaliz| by a path to it.

Options can be long options starting with \verb|--| or short options, a single letter or number prefixed by \verb|-|. Short options can be bundled into a string. A special case is \verb|-x|, setting the parallelization. The order of options and \verb|<project>| is irrelevant. 

Examples, assuming that the Normaliz directory is the working directory and \verb|normaliz| (or \verb|normaliz.exe|) resides in it, MacOs or Linux:
\begin{Verbatim}
./normaliz -c example/small

./normaliz -c -x=16 example/A553
\end{Verbatim}
For MS Windows the equivalent commad lines are
\begin{Verbatim}
normaliz.exe -c example\small

normaliz.exe -c -x=16 example\A553
\end{Verbatim}

\TopIndex

\SectStart{Input}

The input file contains: (i) the definition of the ambeint space, (ii) the algebraic number field in the case of algebraic polyhedra, (iii) the definition of cones, polyhedra and lattices by generators or constraints, (iv) options for computation goals and algorithmic variants,(v)  numerical parameters for certain computations, (vi) a polynomial for certain computations. One can insert C style comments \verb|\*...*/| between input items.\bigskip

\ContLine{Ambient space}
\ContLine{Rational numbers}
\ContLine{Algebraic numbers}
\ContLine{Vectors}
\ContLine{Matrices}
\ContLine{Generators}
\ContLine{Constraints}
\ContLine{Tabular and symbolic constraints}
\ContLine{Grading and dehomogenization}
\ContLine{Polynomials}
\ContLine{Numerical parameters}
\ContLine{Precomputed types}
\ContLine{Additional input types}
\medskip

Input types can be mixed with some obvious restrictions. Roughly speaking, Normalit forms the cone defined by the generators and intersects it with the cone defined by the constraints. The same applies to lattices. See manual for a more precise description. 

\TopIndex

\SubSectStart{Ambient space}\index{ambient space}
The first line of the input file sets the dimension of the ambient space:
\begin{Verbatim}
amb_space <d>
\end{Verbatim}
or
\begin{Verbatim}
amb_space auto
\end{Verbatim}
\verb|auto| is only allowed if the first item, for which the dimension must be known, is a formatted vector or matrix.

\SubSectStart{Rational numbers}\index{rational number}\index{number!rational}

All stsndard formats can be used: integers, fractions, decimal fractions, standard floating point notation. Some input types accept only integral numbers:
\begin{Verbatim}
lattice     cone_and_lattice   offset         open_facets
congruences inhom_congruences  rees_algebra   lattice_ideal
grading     dehomogenization   signs          strict_signs
\end{Verbatim}

\SubSectStart{Algebraic numbers}\index{algebraic number}\index{number!algebraic}

For algebraic polyhedra the definition of the number field must follow the definition of the ambeint space:
\begin{Verbatim}
	number_field min_poly (<poly>) embedding [<center> +/- <radius>]
\end{Verbatim}
\verb|<poly| is a polynomial with rational coefficients (integers or fractions) in one indeterminate, the field generator. The latter is named by a single letter different from \verb|e| and verb|x|. The zero of the minimal polynomial is located in the interval \verb|<center>| $\pm$ \verb|<radius>|.

An algebraic number is a sum of terms, and each term is of type \verb|<coeff>[*]<gen>^<exp>| with the usual simplifications for the exponents $0$ and $1$.

The following input types are NOT allowed for algebraic polytopes:
\begin{center}
\texttt{
\begin{tabular}{llll}
lattice &strict\_inequalities&strict\_signs&open\_facets\\
cone\_and\_lattice& inhom\_congruences& lattice\_ideal&offset\\
congruences& hilbert\_basis\_rec\_cone &rees\_algebra & rational\_lattice\\
rational\_offset
\end{tabular}
}
\end{center}

\EndOfPage


\SubSectStart{Vectors}\index{vector}

A plain vector is given by\index{plain vector}\index{vector!plain}
\begin{Verbatim}
<T>
<x>
\end{Verbatim}
\ttt{<T>} denotes the type and \ttt{<x>} is the vector itself The entries are separated by spaces. The number of components is determined by the type of the vector and the dimension of the ambient space. It can be sparse, given by\index{sparse vector}\index{vector!sparse}
\begin{Verbatim}
sparse <entries>;
\end{Verbatim}
where \verb|<entries>| is a sequence  of pairs \verb|<c>:<v>|. In each pair \verb|<c>|is the index of a coordinate and \verb|<v>| is its value. \verb|<c>| can also stand for a range of coordinates in the form \verb|<first>..<last>|. The concluding semicolon is mandatory.

A formatted vector is given by\index{formatted vector}\index{vector!formatted}
\begin{Verbatim}
<T>
[<x>]
\end{Verbatim}
where \verb|<x>| is a sequence of entries separated by spaces, commas or semicolons.

A special vector is \index{unit vector}\index{vector!unit}
\begin{itemize}
	\itemtt[unit\_vector <n>] represents the $n$-th unit vector in $\RR^d$ where $n$ is the number given by \verb|<n>|.
\end{itemize}

 
\EndOfPage

\SubSectStart{Matrices}\index{matrix}

A plain matrix is built as follows:\index{plain matrix}\index{matrix!plain}
\begin{Verbatim}
<T> <m>
<x_1>
...
<x_m>
\end{Verbatim}
\verb|<T>| is the type and \verb|<m>| the number of rows, the latter given by \verb|<x_1>...<x_m>|. The number of columns is defined by the value of \verb|amb_space| and the type.

The matrix can be transposed:\index{transpose matrix}\index{matrix!transpose}
\begin{Verbatim}
<T> transpose <c>
<x_1>
...
<x_m>
\end{Verbatim}
with \verb|<c>| as the number of entries of each row of the input  and \verb|<x_1>...<<x_m>| are the columns of the resulting matrix. The number of rows of the input is calculated from  the value of \verb|amb_space| and the type.

Both matrices and transposed matrices can be sparse. The keyword \verb<sparse> follows \verb|<m>| or \verb|<c>|. 

A formatted matrix is built as follows:\index{formatted matrix}\index{matrix!formatted}
\begin{Verbatim}
<T>
[ [<x_1>]
...
[<x_m>] ]
\end{Verbatim}
It can also be transposed. 

The unit matrix can be given to every input type that expects a matrix:\index{unit matrix}\index{matrix!unit}
\begin{itemize}
	\itemtt[unit\_matrix]
\end{itemize}
The number of rows is defined by \verb|amb_space| and the type of the matrix, as usual.

\EndOfPage

\SubSectStart{Generators}\index{generators}

\begin{itemize}
	\itemtt[cone] is a matrix with $d$ columns. Every row represents a vector, and they define the cone generated by them.
	
	\itemtt[subspace] is a matrix with $d$ columns. The linear subspace generated by the rows is added to the cone.
	
	\itemtt[polytope] is a matrix with $d-1$ columns. It is internally converted to \verb|cone| extending each row by an entry $1$.This input type automatically sets \verb|NoGradingDenom| and defines the grading $(0,\dots,0,1)$. Not allowed in combination with inhomogeneous types.
	
	\itemtt[cone\_and\_lattice] The vectors of the matrix with $d$ columns define both a cone and a lattice.
	
	If \verb|subspace| is used in combination with \verb|cone_and_lattice|, then the sublattice generated by its rows is added to the lattice generated by \verb|cone_and_lattice|.

	\itemtt[lattice] is a matrix with $d$ columns. Every row represents a vector, and they define the lattice generated by them.
	
	\itemtt[vertices] is a matrix with $d+1$ columns. Each row $(p_1,\dots,p_d,q)$, $q>0$, specifies a generator of a polyhedron (not necessarily a vertex), namely
	$$
	v_i=\biggl(\frac{p_{1}}{q},\dots,\frac{p_{n}}{q}\biggr), \qquad p_i\in\QQ,q\in\QQ_{>0},
	$$
	
	\textbf{Note:}\enspace \verb|vertices| \verb|cone| and \verb|subspace| together define a polyhedron. If \verb|vertices| is present in the input, then the default choices for \verb|cone| aand \verb|subspace| are the empty matrices.
\end{itemize}


\EndOfPage

\index{constraints}
\SubSectStart{Constraints}

Homogeneous constraints:\index{homogeneous constraints}\index{constraints!homogeneous}

\begin{itemize}
	\itemtt[inequalities] is a matrix with $d$ columns. Every row $(\xi_1,\dots,\xi_d)$ represents a homogeneous inequality
	$$
	\xi_1x_1+\dots+\xi_dx_d\ge 0
	$$
	for the vectors $(x_1,\dots,x_d)\in\RR^d$. 
	
	\itemtt[nonnegative] represents a system of inequalities cutting out the positive orthant.
	
	\itemtt[equations] is a matrix with $d$ columns. Every row $(\xi_1,\dots,\xi_d)$ represents an equation
	$$
	\xi_1x_1+\dots+\xi_dx_d= 0
	$$
	for the vectors $(x_1,\dots,x_d)\in\RR^d$.
	
	\itemtt[congruences] is a matrix with $d+1$ columns. Each row $(\xi_1,\dots,\xi_d,c)$ represents a congruence
	$$
	\xi_1z_1+\dots+\xi_dz_d\equiv 0 \mod c, \qquad \xi_i,c\in\ZZ,
	$$
	for the elements $(z_1,\dots,z_d)\in\ZZ^d$.
\end{itemize}\bigskip	

Inhomogeneous constraints:\index{inhomogeneous constraints}\index{constraints!inhomogeneous}

\begin{itemize}
	\itemtt[inhom\_inequalities] is a matrix with $d+1$ columns. We consider inequalities
	$$
	\xi_1x_1+\dots+\xi_dx_d\ge \eta,
	\qquad\text{equivalently,}\qquad
	\xi_1x_1+\dots+\xi_dx_d+(-\eta) \ge 0,
	$$
	represented by the input vectors $	(\xi_1,\dots,\xi_d,-\eta)$.

	
	\itemtt[inhom\_equations] is a matrix with $d+1$ columns. We consider equations
	$$
	\xi_1x_1+\dots+\xi_dx_d= \eta,
	\qquad\text{equivalently,}\qquad
	\xi_1x_1+\dots+\xi_dx_d+(-\eta) = 0,
	$$
	 represented by the input vectors $(\xi_1,\dots,\xi_d,-\eta)$.

	
	\itemtt[inhom\_congruences] We consider a matrix with $d+2$ columns. Each row $(\xi_1,\dots,\xi_d,-\eta,c)$ represents a congruence
	$$
	\xi_1z_1+\dots+\xi_dz_d\equiv \eta \mod c, \qquad \xi_i,\eta,c\in\ZZ,
	$$
	for the elements $(z_1,\dots,z_d)\in\ZZ^d$.
	
\end{itemize}

If there are no cone generators or inequalities, Normaliz automatically assumes the sign inequalities defining the positive orthant positive orthant. This behavior can be broken by an empty matrix \verb|inequalities 0|.

\EndOfPage

\SubSectStart{Tabular and symbolic constraints}\index{constraints!tabular}
\index{tabular constraints}

\begin{itemize}
	\itemtt[constraints <n>] allows the input of \verb|<n>| equations, inequalities and congruences in a format that is close to standard notation. As for matrix types the keyword \verb|constraints| is followed by the number of constraints. If $(\xi_1,\dots,\xi_d)$ is the vector on the left hand side and $\eta$ the number on the right hand side, then the constraint defines the set of vectors $(x_1,\dots,x_d)$ such that the relation
	$$
	\xi_1x_1+\dots +\xi_dx_d \texttt{ rel } \eta
	$$
	is satisfied, where \verb|rel| can take the values $=,\leq,\geq,<,>$ with the represented by input strings \verb|=,<=,>=,<,>|, respectively.
	
	A further choice for \verb|rel| is \verb|~|. It represents a congruence $\equiv$ and requires the additional input of a modulus: the right hand side becomes $\eta (c)$. It represents the congruence
	$$
	\xi_1x_1+\dots \xi_dx_d \equiv \eta \pmod c.
	$$

	A right hand side $\neq 0$ makes the input inhomogeneous, as well as the relations $<$ and $>$. Strict inequalities are always understood as conditions for integers. So
	$$
	\xi_1x_1+\dots +\xi_dx_d < \eta
	$$
	is interpreted as
	$$
	\xi_1x_1+\dots \xi_dx_d \le \eta-1.
	$$

	\itemtt[constraints <n> symbolic]\index{symbolic\index{constraints!symbolic} constraints} where \verb|<n>| is the number of constraints in symbolic form that follow.

Symbolic constraints are given in traditional mathematical form  Note that every symbolic constraint (including the last) must be terminated by a semicolon.The left and right hand side can be affine-linear expressions in the coordinates given by \verb|x<[i>]| where \verb|<i>| varies between $1$ and $d$. 

The interpretation of homogeneity follows the same rules as for tabular constraints.

\end{itemize}

\SubSectStart{Grading and dehomogenization}

\begin{itemize}
	\itemtt[grading] is a vector of length $d$ representing the linear form that gives the grading. (Speciasl rules for lattice ideal input)
	
	\itemtt[total\_degree] is the grading with all coordinates equal to $1$.
	
	\itemtt[dehomogenization] is a vector of length $d$ representing the linear form that gives the dehomogenization.
\end{itemize}

\EndOfPage



\SubSectStart{Polynomials}

For the computation of weighted Ehrhart series and integrals Normaliz needs the input of a polynomial with rational coefficients:
\begin{itemize}
	\itemtt[polynomial <poly expression>]	
\end{itemize}
The polynomial is first read as a string. For the computation the string is converted by the input function of CoCoALib. Therefore any string representing a valid CoCoA expression is allowed. However the names of the indeterminates are fixed: \verb|x[1]|,\dots,\verb|x[<d>| where \verb|dN>]| is the value of \verb|amb_space|. The polynomial must be concluded by a semicolon.

\SubSectStart{Numerical parameters}\index{numerrical parameters}

\begin{itemize}
	\itemtt[expansion\_degree <n>] where \verb|<n>| is the number of coefficients in a series expansion to be computed and printed.
	
	\itemtt[nr\_coeff\_quasipol <n>] where \verb|<n>| is the number of highest coefficients in a quasipolynomial to be printed.
	
	\itemtt[face\_codim\_bound <n>] where \verb|<n>| is the bound for the codimension of the faces to be computed.
	
	\itemtt[decimal\_digits <n>] where \verb|<n>| sets the precision to $10^{-n}$ (for computation with signed decomposition).
	
	\itemtt[block\_size\_hollow\_tri <n>] sets the block size for distributed computation to \verb|<n>|.

\end{itemize}

\SubSectStart{Precomputed types}\label{precomputed types}\index{precomputed types}

Precomputed types are used for the recycling of data from previous  computations, namely extreme rays and support hyperplanes and the coordinate transformations represented by the generated sublattice (or subspace) and the maximal subspace contained in  the cone. The latter are only required if they are different from the default vaues $\ZZ^d$ (or $\RR^d$) and $\{0\}$, respectively.

\begin{itemize}
	\itemtt[extreme\_rays] is a matrix with $d$ columns.
	
	\itemtt[maximal\_subspace] is a matrix with $d$ columns.
	
	\itemtt[generated\_lattice] is a matrix with $d$ columns.
	
	\itemtt[support\_hyperplanes] is a matrix with $d$ columns.
	
\end{itemize}

Further admitted types for precomputed data: \ttt{grading}, \ttt{dehomogenization}.

\EndOfPage

\SubSectStart{Additional input types}

\begin{itemize}

	\itemtt[rees\_algebra] is a matrix with $d-1$ columns. It is internally converted to type \verb|cone| in two steps: (i) each row is extended by an entry $1$ to length $d$. (ii) The first $d-1$ unit vectors of length $d$ are appended. Not allowed in combination with inhomogeneous types.
	
	\itemtt[rational\_lattice] is a matrix with $d$ columns. Its entries can be fractions. Every row represents a vector, and they define the sublattice of $\QQ^d$ generated by them.

	\itemtt[saturation] is a matrix with $d$ columns. Every row represents a vector, and they define the lattice $U\cap \ZZ^d$ where $U$ is the subspace generated by them. (If the vectors are integral, then $U\cap \ZZ^d$ is the saturation of the lattice generated by them.)
	
	\itemtt[signs] is a vector with $d$ entries in $\{-1,0,1\}$.
	It stands for a matrix of type \verb|inequalities| composed of the sign inequalities $x_i\ge 0$ for the entry $1$ at the $i$-th component and the inequality $x_i\le 0$ for the entry $-1$. The entry $0$ does not impose an inequality.
	
	\itemtt[excluded\_faces] is a matrix with $d$ columns. Every row $(\xi_1,\dots,\xi_d)$ represents an inequality
	$$
	\xi_1x_1+\dots+\xi_dx_d> 0
	$$
	for the vectors $(x_1,\dots,x_d)\in\RR^d$. It is considered as a homogeneous input type though it defines inhomogeneous inequalities. The faces of the cone excluded by the inequalities are excluded from the Hilbert series computation, but \verb|excluded_faces| behave like \verb|inequalities| in almost every other respect.
	
	\itemtt[offset] is a vector with $d$ integer entries. It defines the origin of the affine lattice.
	
	\itemtt[rational\_offset] is a vector with $d$ rational entries. It defines the origin of the rational affine lattice.
	
	\itemtt[strict\_inequalities] is a matrix with $d$ columns. We consider inequalities
	$$
	\xi_1x_1+\dots+\xi_dx_d\ge 1,
	$$
	represented by the input vectors $	(\xi_1,\dots,\xi_d)$.
	
	\itemtt[strict\_signs] is a vector with $d$ components in $\{-1,0,1\}$. It is the ``strict'' counterpart to \verb|signs|. An entry $1$ in component $i$ represents the inequality $x_i>0$, an entry $-1$ the opposite inequality, whereas $0$ imposes no condition on $x_i$. 
	
	\itemtt[inhom\_excluded\_faces] is a matrix with $d+1$ columns. Every row $(\xi_1,\dots,\xi_d,-\eta)$ represents an inequality
	$$
	\xi_1x_1+\dots+\xi_dx_d> \eta
	$$
	for the vectors $(x_1,\dots,x_d)\in\RR^d$. The faces of the polyhedron excluded by the inequalities are excluded from the Hilbert and Ehrhart series series computation, but \verb|inhom_excluded_faces| behave like \verb|inhom_inequalities| in almost every other respect. 
	
	\itemtt[hom\_constraints] for the input of equations, non-strict inequalities and congruences in the same format as \verb|constraints|, except that these constraints are meant to be for a homogeneous computation. It is clear that the left hand side has only $d-1$ entries now. Also allowed for symbolic constraints.
	
	\itemtt[projection\_coordinates] It is a $0$-$1$ vector of length $d$.

	The entries $1$ mark the coordinates of the image of the projection. The other coordinates give the kernel of the projection.
	
	\itemtt[open\_facets] is a vector of length $d$ with entries $\in \{0,1\}$. (See manual)
	
	\itemtt[hilbert\_basis\_rec\_cone] is a matrix with $d$ columns. It contains the precomputed Hilbert basis of the recession cone.
	
	\itemtt[lattice\_ideal] is a matrix with $d$ columns containing the generators of the lattice ideral in the Laurent polynomial ring. 
	
\end{itemize}

\EndOfPage


\SectStart{Computation}

\SubSectStart{Computation goals}

The computation goal \verb|Sublattice| does not imply any other computation goal. Most other computation goals include \verb|Sublattice| and \verb|SupportHyperplanes|. The exceptions are:
\begin{arab}
	\item certain computations based on the dual algorithm see Section~\ref{InhomDual};
	\item \verb|Projection| or \verb|ProjectionFloat| applied to parallelotopes; see Section~\ref{InhomDual};
	\item computations done completely by symmetrization.
\end{arab}
If you are in doubt whether your desired data will be computed, add an explicit computation goal.


\SubSubSectStart{Support hyperplanes and extreme rays}

\begin{itemize}
	\itemtt[SupportHyperplanes, -s] triggers the computation of support hyperplanes and extreme rays.
	
	\itemtt [IntegerHull, -H] computes the integer hull of a polyhedron. Implies the computation of the lattice points in it.

	More precisely: in homogeneous computations it implies \verb|Deg1Elements|, in inhomogeneous computations it implies \verb|HilbertBasis|.
	
	\itemtt[ProjectCone] Normaliz projects the cone defined by the input data onto a subspace generated by selected coordinate vectors and computes the image with the goal \verb|SupportHyperplanes|.
\end{itemize}
	
For the following we only need the support hyperplanes and the lattice:


\begin{itemize}
	\itemtt[IsGorenstein, -G]: is the monoid of lattice points Gorenstein? In addition to answering this question, Normaliz also computes the generator of the interior of the monoid (the canonical module) if the monoid is Gorenstein.
\end{itemize}


\EndOfPage


\SubSubSectStart{Hilbert basis and lattice points}

\begin{itemize}
	
	\itemtt[HilbertBasis, -N] triggers the computation of the Hilbert basis. In inhomogeneous computations it asks for the Hilbert basis of the recession monoid \emph{and} the module generators.
	
	\itemtt[IsIntegrallyClosed, -w]: is the original monoid integrally closed? Normaliz stops the Hilbert basis computation as soon as it can decide whether the original monoid contains the Hilbert basis. Normaliz tries to find the answer as quickly as possible. This may include the computation of a witness, but not necessarily. If you need a witness, use \verb|WitnessNotIntegrallyClosed|.
	
	\itemtt[Deg1Elements, -1] restricts the computation to the degree $1$ elements of the Hilbert basis in homogeneous computations (where it requires the presence of a grading).
	
	\itemtt[LatticePoints] is identical to \verb|Deg1Elements| in the homogeneous case, but implies \verb|NoGradingDenom|. In inhomogeneous computations it is a synonym for \verb|HilbertBasis|.
	
	\itemtt[ModuleGeneratorsOverOriginalMonoid, -M] computes a minimal system of generators of the integral closure over the original monoid. Requires the existence of original monoid generators.
\end{itemize}

\EndOfPage

\SubSubSectStart{Enumerative data, volumes and integrals}

The computation goals in this section require a grading. They include \verb|SupportHyperplanes|.

\begin{itemize}
	\itemtt [HilbertSeries,-q] triggers the computation of the Hilbert series.
	
	\itemtt[EhrhartSeries] computes the Ehrhart series of a polytope, regardless of whether it is defined by homogeneous or inhomogeneous input. In the homogeneous case it is equivalent to \verb|HilbertSeries| + \verb|NoGradingDenom|, but not in the inhomogeneous case.
	
	\itemtt[Multiplicity, -v] restricts the computation to the multiplicity.
	
	\itemtt[Volume, -V] computes the lattice normalized and the Euclidean volume of a polytope given by homogeneous or inhomogeneous input (implies \verb|Multiplicity| in the homogeneous case, but also sets \verb|NoGradingDenom|).	
	
	\itemtt[NumberLatticePoints] finds the number of lattice points in a polytope. They are not stored.
\end{itemize}
The following computation goals ned the input of a polynomial:
\begin{itemize}	
	\itemtt[WeightedEhrhartSeries, -E] makes Normaliz compute a generalized Ehrhart series.

	\itemtt[VirtualMultiplicity, -L] makes Normaliz compute the virtual multiplicity of a weighted Ehrhart series.
	
	\itemtt[Integral, -I] makes Normaliz compute an integral over a polytope. Implies \verb|NoGradingDenom|.
\end{itemize}

\EndOfPage

\SubSubSectStart{Triangulation}

\begin{itemize}
	
	\itemtt[Triangulation, -T] makes Normaliz compute, store and export the full triangulation.
	
	\itemtt[AllGeneratorsTriangulation] makes Normaliz compute and store a triangulation that uses all generators.
	
	\itemtt[LatticePointTriangulation] makes Normaliz compute and store a triangulation that uses all lattice points in a polytope.
	
	\itemtt[UnimodularTriangulation] makes Normaliz compute and store a unimodular triangulation.
	
\end{itemize}

\SubSubSectStart{Face structure}

\begin{itemize}
	\itemtt[FVector] computes  f-vector of a polyhedron.
	
	\itemtt[FaceLattice] computes the set of faces.
\end{itemize}
The range can be restricted by the numerical parameter \verb|face_codim_bound|. 
There are dual cone we have:
\begin{itemize}
	\itemtt[DualFVector]
	\itemtt[DualFaceLattice]
\end{itemize}

\SubSubSectStart{Automorphism groups}
\begin{itemize}
	\itemtt[Automorphisms] computes the integral automorphisms of rational polyhedra and the algebraic automorphisms of algebraic polytopes.
	
	\itemtt [RationalAutomorphisms] computes the rational automorphisms of rational polytopes.
	
	\itemtt[EuclideanAutomorphisms] computes the euclidean automorphisms of rational and algebraic polytopes.
	
	\itemtt [CombinatorialAutomorphisms] computes ate combinatorial automorphisms of polyhedra.
	
	\itemtt[AmbientAutomorphisms] computes automorphisms induce by permutations of coordinates of the ambient space.
	
	\itemtt[InputAutomorphisms] computes taional (or algebraic) automorphisms based solely on the input and initial coordinate transformations.
\end{itemize}


\EndOfPage

\SubSubSectStart{Additional computation goals}

\begin{itemize}
	\itemtt[Sublattice, -S] (upper case S) asks Normaliz to compute the coordinate transformation to and from the efficient sublattice.
	\itemtt[VerticesFloat] converts the format of the vertices to floating point. It implies \texttt{SupportHyperplanes}.
	
	\itemtt[SuppHypsFloat] converts the format of the support hyperplanes to floating point. It implies \texttt{SupportHyperplanes}.
	
	\itemtt[ExtremeRaysFloat] does the same for the extreme rays.
	\itemtt [WitnessNotIntegrallyClosed] With this option, Normaliz stops the Hilbert basis computation as soon it has found a witness confirming that the original monoid is not integrally closed.
	\itemtt[HSOP] lets Normaliz compute the degrees in a homogeneous system of parameters and the induced representation of the Hilbert or Ehrhart series series. Note that \ttt{HSOP} does not imply \ttt{HilbertSeries} or \ttt{EhrhartSeries}.
	\itemtt[NoPeriodBound] This option removes the period bound that Normaliz sets for the computation of the Hilbert quasipolynomial (presently $10^6$).
	\itemtt [ClassGroup, -C] is self explanatory, includes \verb|SupportHyperplanes|. Not allowed in inhomogeneous computations.
	\itemtt[ConeDecomposition, -D] Normaliz computes a disjoint decomposition of the cone into semiopen simplicial cones. Implies \verb|Triangulation|.
	\itemtt[TriangulationSize, -t] makes Normaliz count the simplicial cones in the full triangulation.
	
	\itemtt[TriangulationDetSum] makes Normaliz additionally sum the absolute values of their determinants.
	
	\itemtt[StanleyDec, -y] makes Normaliz compute, store and export the Stanley decomposition.
	
	\itemtt[PlacingTriangulation] combinatorially defined triangulation, see manual.
	\itemtt[PullingTriangulation] ditto.
	
	\itemtt[Incidence] computes the incidence of extreme rays and facets.
	
	\itemtt[DualIncidence] the transpose of \verb|Incidence|.
	
	\itemtt[IsEmptySemiopen] checks whether a semiopen polyhedron is empty.
	
	\itemtt[IsPointed]: is the efficient cone $C$ pointed? This computation goal is sometimes useful to give Normaliz a hint that a nonpointed cone is to be expected. See Section~\ref{IsPointed}.
	
	\itemtt[IsDeg1ExtremeRays]: do the extreme rays have degree~$1$?
	
	\itemtt[IsDeg1HilbertBasis]: do the Hilbert basis elements have degree~$1$?
	
	\itemtt[IsReesPrimary]: for the input type \verb|rees_algebra|, is the monomial ideal primary to the irrelevant maximal ideal?

\end{itemize}

There are several more computation goals that are used internally by Normaliz and the communication with external packages. See manual.

\EndOfPage

\SectStart{Execution}


\printindex

\end{document}